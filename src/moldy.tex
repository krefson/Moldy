\documentstyle{report}
\oddsidemargin=0.5in
\textwidth=6in
\textheight=8.5in
\title{{\Huge Moldy User's Manual}}
\author{Keith Refson\\Department of Earth Sciences\\Parks Road
\\Oxford OX1 3PR\\keith@uk.ac.ox.earth\\}

\newcommand{\moldy}{{\em Moldy}}
\newcommand{\etc}{{\em etc}}
\newcommand{\eg}{{\em e.g}.\ }
\newcommand{\ie}{{\em i.e}.\ }
\newcommand{\bm}[1]{\mbox{\boldmath \protect\(#1\protect\)}}

\newcommand{\erf}{\mbox{erf}}
\newcommand{\erfc}{\mbox{erfc}}
\begin{document}
\maketitle
\tableofcontents

\chapter{Introduction}

\moldy\  is a computer program for performing molecular dynamics
simulations of condensed matter.  It can handle any assembly of rigid
polyatomic molecules, atoms or ions and any mixture thereof. It uses
the `link cell' method to calculate short-range forces and the Ewald
sum technique to handle long-range electrostatic forces.  Simulations
may be performed either in the usual $NVE$ ensemble or in an $N\sigma
T$ ensemble using the constant-stress method of Parinello and Rahman.
As the MD cell need not be cubic, the program is equally suitable for
simulations of solids and liquids.

Most existing MD programs are limited in their capabilities, for
example to one kind of potential function, or molecular symmetry, or
to some restricted number of molecules.  \moldy\  is (as far as
possible) free from such arbitrary constraints.  The system is
specified at the beginning of each run and its size is only limited by
the amount of memory available to the program: if a system is too
large to handle, the solution is to buy some more memory.  The system
may contain a mixture of an arbitrary number of molecular species,
each with an arbitrary number of atoms and a arbitrary number of
molecules of each. Molecules or ions may be monatomic or polyatomic,
linear or three dimensional in any combination.  The potential
functions may be of the Lennard-Jones, Buckingham (including
Born-Meyer) or MCY types, and other potential types may be easily
added.  Such flexibility is possible because \moldy\  is written in the
`C' language which permits dynamic memory allocation.

\moldy\  is written to be highly portable and has been tested on a wide
range of computers and operating  systems, including CRAY COS, VAX/VMS
and  Unix(tm) (both   BSD  and  system  V  varieties).   It should  be
straightforward to  move  it to  any  other machine  with a   good `C'
compiler.

To be of real use a simulation  program must run efficiently on modern
high-speed computers, which  are  increasingly of vector   or parallel
architectures.  \moldy\  is written so  as to be highly vectorisable and
has been  tested on a   range of  vector machines  from  manufacturers
including Cray, Convex, Stardent and  Alliant.  On the cray XMP-48 its
performance can exceed 100 MFlop/sec (on a suitably large  system).  A
experimental version for parallel  machines with shared memory is also
available.

\chapter{Algorithms and Equations}

This chapter  describes the implementation  of the  molecular dynamics
technique as used in \moldy.  It is not intended as an introduction to
MD  simulations, and  does assume some    familiarity  with the  basic
concepts of microscopic models  and simulations thereof.  The book  by
Allen and Tildesly  \cite{allen:87}  is a  very good    introductory text,
covering   both the  theory  and  the  practicalities and  is   highly
recommended.  It   also  contains   comprehensive  references  to  the
scientific literature of microscopic computer simulation.

\moldy\  is designed to simulate a common class of models of atomic or
molecular systems. The assumptions are: that the system is an assembly
of  {\em rigid   molecules},  atoms   or  ions;  that  the  forces  of
interaction are derived   from {\em  continuous  potential  functions}
acting between (usually atomic) {\em sites} on each molecule; that the
dynamics are governed by the {\em classical} Newton-Euler equations of
motion.  A major aim of \moldy\ has been to allow the most  general of
models within that class and to impose as few restrictions as possble.
In particular arbitary mixtures of different molecules are allowed
and several popular forms of potential functions are catered for.

\section{The Equations of Motion}
The forces determining the dynamics of the system are derived from the
potential    function    denoted     by   $\phi_{i\alpha  j\beta}(\bm{r}
_{i\alpha j\beta})$.  The indices $i$ and $j$ run  over all molecules
in the system and  $\alpha$ and $\beta$ over  sites on  the respective
molecule\footnote{A comment on notation is appropriate here.  In this
chapter, site quantities are denoted by {\em lowercase} letters,
molecular quantities by {\em uppercase}, sites are indexed by {\em greek}
letters and molecules by {\em roman}. A missing index denotes
a sum over the corresponding sites or molecules so that, for example
$\bm{r}_{i\alpha j\beta}$ is a site-site vector and $\bm{F}_{ij}$ the
molecule-molecule force}.  
In that case the total potential energy of the system, $U$, is
\begin{equation}
U = \sum_i \sum_{j > i} \sum_\alpha \sum_\beta \phi_{i\alpha j\beta}(
\bm{r}_{i\alpha j\beta}).
\end{equation}
The total force on molecule $i$ is
\begin{equation}
\label{eqn:comf}
\bm{F}_i = \sum_j \sum_\beta \sum_\alpha \bm{f}_{i\alpha j\beta}
\end{equation}
where $\bm{f}_{i\alpha j\beta} = - \bm{\nabla} \phi_{i\alpha  j\beta}(
\bm{r}_{i\alpha j\beta})$ is the force acting on site $\beta$ of molecule
$j$ from site $\alpha$ of molecule $i$.  A torque
\begin{equation}
\label{eqn:comt}
\bm{N}_i = \sum_\alpha (\bm{r}_{i \alpha} - \bm{R}_{i}) 
\bm{\times f}_{i\alpha} 
\end{equation}
also acts on molecule $i$.  $\bm{R}_{i} = 1/M_i \sum_\alpha
m_{i\alpha} \bm{r}_{i\alpha}$ is the centre of mass of molecule $i$.

The motion is governed by the Newton-Euler equations
\begin{equation}
M_i\ddot{\bm{R}_i} = \bm{F}_i \label{eqn:newton}
\end{equation}
\begin{equation}
\bm{I_i \cdot} \dot{\bm{\omega}}_i - \bm{\omega_i \times I_i \cdot
\omega_i} = \bm{N_i} \label{eqn:euler}
\end{equation}
where  $\bm{\omega}_i$ is the angular velocity of the molecule, 
$ \bm{I}_i = \sum_{\alpha} m_{i\alpha} 
( p_{i\alpha}^2 \bm{1} - \bm{p}_{i\alpha}\bm{p}_{i\alpha} ) $ 
is the inertia tensor and 
$\bm{p}_{i\alpha} = \bm{r}_{i\alpha} - \bm{R}_i $
is the atomic site co-ordinate relative to the molecular centre of mass.

% Quaternions
\label{sec:quaternions}
The orientations of the molecules are represented by {\em quaternions}
as  has now  become common  practice.   They are  preferred over Euler
angles for two reasons. Firstly they lead to equations of motion which
are  free  of singularities   \cite{evans:77} which   means  that   no
special-case  code is required. This  leads to much improved numerical
stability      of  the  simulation  \cite{evans:77b}.   Secondly,
molecular symmetry   operations   and  combinations of  rotations  are
elegantly expressed   in    terms  of     a     simple      quaternion
algebra\cite{evans:77b,pawley:85b}.

A quaternion  is an ordered number  quartet  which obeys  the  algebra
given  by   Pawley\cite{pawley:81}.  The multiplication   rule in that
reference may be restated as a matrix product treating each quaternion
as a  4-vector.  If  ${\bf p} \equiv (p_0,p_1,p_2,p_3)$ and  ${\bf q}
\equiv  (q_0,q_1,q_2,q_3)$  are quaternions then
\begin{equation}
{\bf pq} = 
\left(
\begin{array}{rrrr}
p_0 & -p_1 & -p_2 & -p_3 \\
p_1 &  p_0 & -p_3 &  p_2 \\
p_2 &  p_3 &  p_0 & -p_1 \\
p_3 & -p_2 &  p_1 &  p_0 
\end{array}
\right) \left(
\begin{array}{r}
q_0 \\ q_1 \\ q_2 \\ q_3
\end{array}
\right)
\end{equation}
The quaternion $ {\bf \tilde{q}} $ conjugate to ${\bf q}$ is defined
as ${\bf \tilde{q}} = (q_0,-q_1,-q_2,-q_3)$ so that
\begin{equation}
{\bf q\tilde{q}} = (q_0^2+q_1^2+q_2^2+q_3^2,0,0,0) .
\label{eqn:qnorm}
\end{equation}
The {\em norm} is defined as $ |{\bf q}| \equiv
\sqrt{q_0^2+q_1^2+q_2^2+q_3^2} $ and {\bf q} is called a {\em unit}
quaternion if $ |{\bf q}| = 1 $ .  

Any possible rotation can represented by a unit quaternion. Du Val shows
\cite{duval:64} that if $ {\bf q} = (\cos \alpha/2, \bm{l} \sin
\alpha/2) $ (where we have combined the last three components to form
a 3-vector) and ${\bf p} = (0,\bm{r})$ then the operation
\begin{equation}
{\bf p'} \equiv (0,\bm{r'}) = {\bf q p \tilde{q}}
\end{equation}
corresponds to a rotation of the vector $\bm{r}$ by an angle of
$\alpha$ about the axis $\bm{l}$.  The components may also be expressed in
terms of the Euler angles as
\footnote{The definition of quaternions used here differs from that used
in Evans' paper \cite[equation 21]{evans:77} in the sign of $q_2$ or
$\xi$.  This error has been compounded by subsequent authors
\cite{sonnenschein:85,smith:82,laakonsen:85} who also managed to 
permute the components which means that the parameters do not form an
ordered number quartet which obeys quaternion algebra. Like Allen and
Tidesley \cite[page 88]{allen:87} we follow the definition of of
Goldstein \cite[pages 143 and 155]{goldstein:80}.}
\begin{eqnarray}
q_0 & = & \cos \frac{\phi+\psi}{2} \cos \frac{\theta}{2} \nonumber \\
q_1 & = & \sin \frac{\phi-\psi}{2} \sin \frac{\theta}{2} \nonumber \\
q_2 & = & \cos \frac{\phi-\psi}{2} \sin \frac{\theta}{2} \nonumber \\
q_3 & = & \sin \frac{\phi+\psi}{2} \cos \frac{\theta}{2}.
\end{eqnarray}

The relationship between the time derivative of a quaternion and the
principal frame angular velocity was given by Evans \cite[Equation
27]{evans:77} and rewritten using quaternion algebra by Refson
\cite{refson:87a} as
\begin{equation}
2 {\bf \dot{q}} = {\bf q}(0,\bm{\omega}_p)
\label{eqn:qomega}
\end{equation}
The second derivative is given by
\begin{eqnarray}
2{\bf \ddot{q}} & = & {\bf q}(- 1/2 \omega_p^2,\dot{\bm{\omega}_p}) \\
 & = & {\bf q}(-2 |{\bf \dot{q}}|^2,\dot{\bm{\omega_p}}) \label{eqn:qddot}
\end{eqnarray}
Equations \ref{eqn:qddot} and \ref{eqn:euler} allow the simulation to
be implemented using quaternions and their derivatives as the dynamic
variables for rotational motion, and this is the method employed in
\moldy.  This second order formulation was first used by Powles {\em et
al}.\ \cite{powles:79} and Sonnenschein showed\cite{sonnenschein:85}
that it gives substantially better stability than if angular
velocities and accelerations are used as dynamic variables.
 
Using equations~\ref{eqn:qddot} to describe the dynamics means that
they are integrated as if all four components were independent.
Therefore the normalisation condition $\bf q \tilde{q} = 1$ may not be
exactly satisfied after performing an integration step.  \moldy\
adopts the usual practice of scaling all components of the quaternion
after each timestep to satisfy the normalisation condition
\cite{evans:77b}.

It is less widely realised that the second order equations
(\ref{eqn:qddot}) intoduce a {\em second} ``unconstrained'' variable
into the procedure.  Differentiating equation~\ref{eqn:qnorm} gives a
constraint on the quaternion derivatives
\begin{equation}
q_0\dot{q_0} + q_1\dot{q_1} + q_2\dot{q_2} + q_3\dot{q_3} = 0
\label{eqn:qconst} 
\end{equation}
which is just the $q_0$ component of equation~\ref{eqn:qomega}.
Just as with the normalisation condition, the integration algorithm
will not preserve this condition exactly unless explicit measures are
taken.  After each timestep the constraint may be re-established by
subtracting the discrepancy from the quaternion derivatives.  If
$\delta = q_0\dot{q_0} + q_1\dot{q_1} + q_2\dot{q_2} + q_3\dot{q_3}$
then the corrected quaternion derivatives are given by
\begin{equation}
{\bf \dot{q}^\prime } =  {\bf \dot{q}} - \delta {\bf q} .
\label{eqn:qconcorr}
\end{equation}
Experiments conducted while developing \moldy\ show that enforcing
this constraint significantly decreases the fluctuations in the total
energy. 

Linear molecules are a slightly special case as the moment of inertia
about the molecular axis is zero.  Though there are unique methods to
represent this situation \cite[page 90]{allen:87} \moldy\ uses a minor
modification of the quaternion algorithm.  All that is necessary is a
little special-case code to avoid dividing by the zero component of
inertia in the solution of equation~\ref{eqn:euler} and to hold the
components of angular velocity and acceleration about the molecular
axis to zero.  This has the considerable advantage of uniform
treatment of all kinds of molecules which is convenient when dealing
with heterogeneous mixtures.

\section{Integration Algorithms}

The dynamical equations~\ref{eqn:newton} and~\ref{eqn:euler} are
integrated using this author's modification \cite{refson:85} of the
Beeman algorithm \cite{beeman:76}.  For atomic systems it is
equivalent to the commonly used Verlet algorithm \cite{verlet:67} but
is preferable for polyatomic molecules because it gives velocities
accurate to order $\delta t^3$ compared to order $\delta t^2$ for the
Verlet algorithm.  This not only permits a more accurate determination
of the kinetic energy, pressure and other dynamic quantities, but is
essential if the (generalised) velocities enter the dynamical
equations themselves.  

Velocity-dependent forces occur in equations~\ref{eqn:euler}
and~\ref{eqn:qddot} and in the Parinello-Rahman constant-pressure
equations (section~\ref{sec:const-stress}). These usually present a
problem to non ``predictor-corrector'' algorithms which are based on
the assumption that the forces depend only on the co-ordinates.
Fincham has devised a scheme to allow integration of the rotational
equations using Verlet-like algorithms \cite{fincham:81}, which is
widely used despite the potential problems caused by the low
accuracy of the velocities being propagated into the dynamics.  The
case of the Parinello-Rahman equations has been handled by
reformulating them to avoid the velocity-dependence \cite{dove:??}.

However both cases are easily and accurately handled by the
modification to Beeman's equations proposed by the author
\cite{refson:85}.  These may be summarised using the symbol $x$ 
to represent any dynamic variable (centre-of-mass co-ordinate,
qauternion or MD cell edge), $\dot{x}^{(p)}$ and $\dot{x}^{(c)}$ to
represent ``predicted'' and ``corrected'' velocities respectively.

\newlength{\myboxw}
\addtolength{\myboxw}{\textwidth}
\addtolength{\myboxw}{-15pt}
\begin{minipage}{\myboxw}
\begin{eqnarray}
& &
\begin{array}{llll}
\makebox[0.5in][l]{\em i} & x(t+\delta t) &=& x(t) + \delta t \, \dot{x}(t) + 
\frac{\delta t^2}{6} 
\left [ 4 \ddot{x}(t) - \ddot{x}(t-\delta t) \right ] \\
\makebox[0.5in][l]{\em ii}  & \dot{x}^{(p)}(t+\delta t) & = & \dot{x}(t) + 
\frac{\delta t}{2} \left [ 3 \ddot{x}(t)- \ddot{x}(t-\delta t) \right ]
\end{array} \nonumber \\
& &
\left .
\begin{array}{llll}
\makebox[0.5in][l]{\em iii}  & \ddot{x}(t+\delta t) & = & F(\{ x_i(t+\delta t),
\dot{x}_i^{(p)}(t+\delta t)\}, i =  1\ldots n) \\
\makebox[0.5in][l]{\em iv} & \dot{x}^{(c)}(t+\delta t) & = & \dot{x}(t) + \frac{\delta t}{6} 
\left [ 2 \ddot{x}(t+\delta t) + 5 \ddot{x}(t)- \ddot{x}(t-\delta t) \right ]\\
\makebox[0.5in][l]{\em v} & 
\lefteqn{\mbox{Replace } \dot{x}^{(p)} \mbox{ with }
\dot{x}^{(c)} \mbox{ and goto iii.  Iterate to convergence}}
\end{array} 
\right \}
\end{eqnarray}
\end{minipage}
\vspace{10pt}

The predictor-corrector cycle of steps {\em iii} to {\em v} are
iterated until the predicted and corrected velocities have converged
to a relative precision of better than 1 part in $10^{-7}$, which in
practice takes 2 or 3 cycles.  This iteration is not as inefficient as
it might at first appear as it does {\em not} include the expensive
part of the calculation --- the recalculation of the site forces.
Only the angular accelerations and quaternion second derivatives must
be evaluated at each iteration using equations~\ref{eqn:euler} and
~\ref{eqn:qddot}, and this operation is relatively cheap.

\section{The Ewald Sum}
\label{sec:ewald}
The long range Coulomb interactions are handled using the Ewald Sum
technique in three dimensions \cite[p.\ 156]{berthaut:52,allen:87}.
The electrostatic potential of a system of charges is expressed as a
sum of short-range and long-range contributions.  Each term is written
as a series,the first in real space and the second, obtained by
Fourier transformation using the periodicity of the MD cell, in
reciprocal space.  The expression for the Coulomb energy $U$ is

\begin{eqnarray}
\label{eqn:ewald}
U & = &  \underbrace{\frac{1}{4 \pi \epsilon_0}
\sum_{\bm{n}} \sum_{i=1}^{N} \sum_{j=i+1}^{N} q_iq_j
\frac{\erfc( \alpha | \bm{r}_{ij} +  \bm{n} |)}{ | \bm{r}_{ij} +
\bm{n} | }}_{\mbox{Real-space term}} 
\nonumber \\
 & + & \underbrace{\frac{1}{2 \epsilon_0 V} \sum_{\bm{k} \ne 0} \frac{1}{k^2}
       e^{-\frac{k^2}{4 \alpha^2}} 
\left \lbrace 
\left | \sum_{i=1}^{N} q_i \cos(\bm{k.r}_i) \right |^2 + 
\left | \sum_{i=1}^{N} q_i \sin(\bm{k.r}_i) \right |^2 
\right \rbrace}_{\mbox{Reciprocal-space term}}\\
 & - & 
\underbrace{\frac{\alpha}{4 \pi^\frac{3}{2} \epsilon_0} 
\sum_{i=1}^{N} q_i^2}_{\mbox{Point self-energy}} 
 - \underbrace{\frac{1}{4 \pi \epsilon_0} \sum_{m=1}^M
\sum_{\kappa=1}^{N_m} \sum_{\lambda=\kappa+1}^{N_m} q_{m\kappa} q_{m\lambda}
\frac{\erf( \alpha | \bm{r}_{\kappa\lambda} |)}{ |
\bm{r}_{\kappa\lambda}| }}_{\mbox{Intra-molecular self energy}}
\nonumber \\
& + & \underbrace{ \frac{1}{8 \pi \epsilon_0 V \alpha^2}
                    \left | \sum_{i=1}^N q_i 
		    \right |^2}_{\mbox{charged system term}}
\; \underbrace{\left [ + \frac{1}{6 \epsilon_0 V} 
                    \left | \sum_{i=1}^N q_i \bm{r}_i 
                    \right |^2 \right ]}_{\mbox{surface dipole term}}
\nonumber
\end{eqnarray}

where we are using the notation

\begin{tabular}{ll}
$\bm{n}$	& Lattice vector of periodic array of MD cell images. \\
$\bm{k}$	& Reciprocal lattice vector of periodic array of MD cell images.. \\
$k$		& Modulus of $\bm{k}$. \\
$i,j$		& absolute indices of all charged sites. \\
$m$		& index of molecules. \\
$\kappa,\lambda$ & indices of sites within a single molecule. \\
$N$		& Total number of charged sites. \\
$M$		& Total number of molecules. \\
$N_m$		& Number of sites on molecule $m$. \\
$q_i$		& Charge on absolute site $i$. \\
$q_{m\kappa}$	& Charge on site $\kappa$ of molecule $m$. \\
$\bm{r}_i$	& Cartesian co-ordinate of site $i$. \\
$\bm{r}_{ij}$	& $\bm{r}_j - \bm{r}_i$. \\
$\alpha$	& Real/reciprocal space partition parameter. \\
$V$		& Volume of MD cell.
\end{tabular}

and the force on charge $i$ is given by
\begin{eqnarray}
\label{eqn:ewald-force}
\bm{f}_i &=& - \nabla_{\bm{r}_i} U \nonumber \\
& = &  \underbrace{\frac{1}{4 \pi \epsilon_0}
\sum_{\bm{n}} \sum_{j \neq i}^{N} q_j \left \lbrace 
\frac{\erfc( \alpha | \bm{r}_{ij} +  \bm{n} |)}{ | \bm{r}_{ij} +\bm{n} | } +
\frac{2 \alpha}{\sqrt{\pi}}e^{- \alpha^2 | \bm{r}_{ij} +  \bm{n} |^2}
\right \rbrace \frac{ \bm{r}_{ij} + \bm{n}}{ | \bm{r}_{ij} + \bm{n} |^2}
}_{\mbox{Real-space term}} \nonumber \\
 & + & \underbrace{\frac{1}{\epsilon_0 V} 
\sum_{\bm{k} \ne 0} q_i \frac{\bm{k}}{k^2} e^{-\frac{k^2}{4 \alpha^2}} 
\left \lbrace 
\sin(\bm{k.r}_i) \sum_{j=1}^{N} q_j \cos(\bm{k.r}_j)  + 
\cos(\bm{k.r}_i) \sum_{j=1}^{N} q_j \sin(\bm{k.r}_j) 
\right \rbrace}_{\mbox{Reciprocal-space term}}\\
& \, & \underbrace{\left [ + \frac{q_i}{6 \epsilon_0 V} 
                    \left ( \sum_{j=1}^N q_j \bm{r}_j
                    \right ) \right ]}_{\mbox{surface dipole term}}
\nonumber
\end{eqnarray}

Both the real- and reciprocal-space series (the sums over $\bm{n}$ and
$\bm{k}$) converge fairly rapidly so that only a few terms need be
evaluated.  We define the {\em cut-off} distances $r_c$ and $k_c$ so
that all terms with $| \bm{r}_{ij} +\bm{n} | < r_c$ and $\bm{k} < k_c$
are included.

Notice that the equation~\ref{eqn:ewald} for the energy contains a
correction for the intra-molecular self-energy, whose derivative is
absent from the equation for the forces
(equation~\ref{eqn:ewald-force}).  This term corrects for interactions
between charges on the {\em same} molecule which are implicitly
included in the reciprocal space sum, but are not required in the
rigid-molecule model.  Though the site forces $\bm{f}_i$ do therefore
include unwanted terms these sum to zero in the evaluation of the
molecular centre-of-mass forces and torques (equations~\ref{eqn:comf}
and \ref{eqn:comt}) (by the conservation laws for linear and angular
momentum).
% Self energies
% uniform sheet correction
% Surface dipole term
\section{Pressure and Temperature}
% Virial, site-site & mol-mol correction.
\section{Periodic Boundaries --- the Link Cell Method}
\label{sec:link-cell}
\section{Temperature Scaling}
\label{sec:rescaling}
% Per species and total rescaling
\section{Constant Stress}
\label{sec:const-stress}
\section{Radial Distribution Functions}
% Formulae for multi-component systems.
\section{The Initial Configuration}
\label{sec:skewstart}
\section{Potential Functions}
% LJ, Buckingham (Born-Meyer) & MCY
% distant potential correction
\section{Frameworks}
% Zeolites, clay
% Discussion of models - wrt unbalanced charges.
% Framework is constrained
\section{How Moldy implements all of this}

\chapter{Running Moldy}

The way \moldy\  is invoked depends to some extent on the operating
system, but usually by issuing the command \verb'moldy'\footnote{On
VMS, {\tt moldy} may be defined as a foreign command by {\tt \$ moldy
:== \$mydisk:[mydir]moldy} For Unix(tm) the executable file {\tt
moldy} should be placed in the shell search path (\eg in the current
directory). For COS the executable binary should be in a local dataset
called {\tt MOLDY}, which is often a permanent dataset, locally
accessed by {\tt ACCESS,DN=MOLDY.}}.  There are two optional arguments
- the name of the control file (see section~\ref{sec:control}) and the
output file (see section~\ref{sec:output}).  If either is omitted,
control input is read from the ``standard input'' which may be a
terminal or a job command file depending on the operating system and
circumstances, and the output is written to ``standard output'' which
may be a terminal or batch job logfile\footnote{Some operating systems
(Unix and COS) allow {\em file redirection} whereby the standard input
is associated with some file.  This may also be used to supply the
control file, provided that no command line parameter is given.}.
Here are examples for VAX/VMS, Unix (tm) and COS (Cray Operating
System), which assume that in each case the command has been set up to
invoke \moldy. Under VMS the commands
\begin{verbatim}
        $ moldy control.dat output.lis
        $ moldy control.dat
\end{verbatim}
will start \moldy\  which will read its input from \verb'control.dat'.
The output will be directed to the file \verb'output.lis' in the first
case and written to the terminal or batch log in the second.  Under
UNIX any of
\begin{verbatim}
        % moldy < control > output.lis
        % moldy control output.lis
        % moldy control
\end{verbatim}
will cause moldy to read from the file called \verb'control' and in
the first two examples to write its output to \verb'output.lis'.
Under COS (Cray Operating System) any of the control statements
\begin{verbatim}
        MOLDY.
        MOLDY,'CONTROL OUTPUT'.
        MOLDY,' < CONTROL > OUTPUT'.
\end{verbatim}
will invoke \moldy\ , which in the first example will read the control
file from the job file following the first \verb'/EOF' marker and send
the output to \verb'$OUT'.  In the last two examples
\moldy\  will read from the local dataset \verb'CONTROL' (which might, for
example, be created using \verb'FETCH' or \verb'ACCESS') and create
another local dataset called \verb'OUTPUT'.

\section{The Control File}
\label{sec:control}

The information needed to initiate and control a run of \moldy\  is
specified in a file known as the {\em control file}. This contains the
parameters governing the run \eg the number of timesteps to be
executed or the frequency of output, and the names of files to be used
\eg for reading a restart configuration from or for writing the
output to. Parameters in the control file are specified by entries of the form
\verb'keyword = value'
which appear one to a line, terminated by the special keyword
\verb'end'.  Spaces and blank lines are ignored as are comments ({\em
i.e.} the remainder of a line following a \verb'#'~symbol) and
keywords may be entered in upper or lower case. For example
\begin{verbatim}
   title= Moldy example      # This is a comment

   # The above blank line is ignored
   nsteps = 1000
   step=0.0005
   restart-file = RESTART.DAT
   end                       # The control file ends here
\end{verbatim}
sets the title of the simulation to ``Moldy example'', the number of
steps to execute to 1000, the timestep to 0.0005ps and supplies the
name of a restart file to read from.

It is not necessary to specify all of the parameters on each run.
Unless it is explicitly assigned a value in the control file, each
parameter has a default value.  This is either the value listed in
table~\ref{tab:parameters} or, in the case where the simulation is
continuing from a restart file, the value it had on the previous run
(see section~\ref{sec:restarting}). Parameters are read in sequence
from the control file, and may appear more than once, in which case
only the final instance is used.  

The two most important parameters are \verb'step' which sets the size
of the simulation timestep (in ps), and \verb'nsteps' which specifies
the number of steps to perform.  Together these control the length of
time to be simulated.  It is also possible to specify that a run
should be terminated after a certain amount of computer time has been
used - given by parameter \verb'cpu-limit'.  This will be particularly
useful in batch mode systems, where the run is killed after some
specified CPU time has elapsed.  Setting \verb'cpu-limit' to the
maximum time allowed will cause \moldy\  to terminate the run {\em
before} the limit is reached and write out a backup file (see
section~\ref{sec:backup}).

There are several kinds of parameters:
\begin{description}

\item[character strings] Apart from \verb'title' these are just file
names \eg \verb'sys-spec-file'\footnote{Note for COS users: There is a
special filename syntax for permanent dataset names ID's and usernames
under COS (as described in the C library reference manual). The forms
allowed are variants of OWN/ID/PDN where OWN is the dataset owner, ID
is the ID string and PDN is the dataset name.  The OWN or ID strings
may be omitted, defaulting to the job owner and null respectively.
Unnecessary slashes may also be omitted so that, for example OWN//PDN
and ID/PDN have the obvious meaning. If only a dataset name is
present, and it is a valid local name, a local dataset is tried first.
Failing that it is taken to be a permanent dataset name.}.  No checks
are performed on the validity of the name (because \moldy\  has to work
on many different computer systems), so if you make a mistake you are
likely to get an error message to the effect that \moldy\  couldn't find
the file you asked for.  To remove a default value, just specify a
null string {\em e.g.} \verb'save-file = '.

\item[booleans] These are just switches which turn a feature off or
on. `0' means off or false and `1' means on or true. The parameters
\verb'text-mode-save', \verb'new-sys-spec', 
\verb'surface-dipole' and \verb'lattice-start' are booleans.

\item[real parameters] Several parameters are real numbers \eg
\verb'step' which specifies the timestep.  They may be entered in the
usual floating point or scientific notation \eg \mbox{\verb'step = 0.0005'}
or \mbox{\verb'step = .5e-3'}, and are taken to be in the units given in
table~\ref{tab:parameters}.

\item[integer parameters] Parameters such as \verb'dump-level' take a
numeric value, which should be an integer.

\item[timestep-related parameters] Several parameters govern when some
calculation begins and ends and how frequently it is performed in
between. These are known as ``begin'', ``end'' and ``interval''
parameters, but are really a special case of integer parameters.  For
example \verb'begin-average', \verb'dump-interval' and
\verb'scale-end'.  The calculation begins {\em on} the timestep
specified on the \verb'begin' parameter, occurs every \verb'interval'
timesteps thereafter and ends {\em after} the timestep specified by
the \verb'end' parameter.  Setting the \verb'interval' parameter to
zero is the usual method of turning that calculation off.

The \verb'begin' and \verb'end' parameters behave in a special fashion
when the simulation is continued from a restart file; they are
interpreted {\em relative} to the current timestep.  Notice especially
that \verb'nsteps', the number of timesteps is treated in this way.

\end{description}

A complete list of the parameters,
their meanings and default values appears in
table~\ref{tab:parameters}. 

\begin{table}
\begin{minipage}{\textwidth}
\caption{Control Parameters}
\label{tab:parameters}
\begin{tabular}{|l|l|l|p{2.7in}|}
\hline
{\bf name} & {\bf type}\footnote{See section \ref{sec:control}} &
{\bf default} & {\bf function} \\ \hline \hline
\verb'title' &	    		character  & {\small Test Simulation} &
A title to be printed on all output. \\
\verb'nsteps' &	    		integer & 			0 &
Number of MD steps to execute. \\
\verb'cpu-limit' &		real &			1e20 &
Terminate run if excessive CPU time used. \\
\verb'step' &	    		real &			0.005 &
Size of timestep \\ \hline
\verb'sys-spec-file' &   	character  &	null &
Name of system specification file.  Appended to control file if null. \\
\verb'lattice-start' &		boolean & 		false  &
Switch for crystalline initial configuration. \\
\verb'save-file' &	   	character  &	null &
File to save restart configuration in. \\
\verb'restart-file' &    	character  &	null &
File to read restart configuration from. \\
\verb'new-sys-spec' &    	boolean & 		false  &
Read restart configuration with changed system specification. \\
\verb'text-mode-save' &  	boolean & 		false &
Write a portable ``restart'' file consisting of control, system
specification and lattice start files. \\ \hline
\verb'density' &	    	real &			1.0 &
Initial density.  Used by {\em skew start} only to determine initial
MD cell dimensions. \\ \hline
\verb'scale-interval' &    	integer & 		10 &
Frequency of velocity scaling to required temperature. \\
\verb'scale-end' &	   	integer & 			1000000 &
When to stop scaling. \\
\verb'scale-options' &           integer & 		0  &
Select variations on scaling. \\
\verb'temperature' &    	real &			0 &
Temperature to set up initial configuration at and to scale to. \\ \hline
\verb'const-pressure' &  	boolean & 		false  &
Whether to use Parinello and Rahman constant stress. \\
\verb'w' &		    	real &			100.0 &
Value of P \& R mass parameter. \\
\verb'pressure' &	    	real &			0 &
Applied pressure. \\ \hline
\verb'alpha' &	    		real &			0.3 &
$\alpha$ parameter for Ewald sum. \\
\verb'k-cutoff' &		real &			2 &
Reciprocal space cut off distance in \AA$^{-1}$. \\
\verb'cutoff' &	    	real &			10 &
Direct space cutoff distance in \AA. \\
\verb'surface-dipole' &	boolean & 		false  &
Include De Leeuw \& Perram term in Ewald sum. \\ \hline
\verb'roll-interval' &   	integer & 		10 &
Period over which to calculate rolling averages. \\
\verb'print-interval' &  	integer & 		10 &
How frequently to print normal output. \\ \hline
\end{tabular}
\end{minipage}
\end{table}
\begin{table}
\begin{minipage}{\textwidth}
\caption{Control Parameters (continued)}
\begin{tabular}{|l|l|l|p{2.7in}|}
\hline
{\bf name} & {\bf type}\footnote{See section \ref{sec:control}} &
{\bf default} & {\bf function} \\ \hline \hline
\verb'begin-average' &   	integer & 		1001 &
When to start accumulating thermodynamic averages. \\
\verb'average-interval' &	integer & 		5000 &
How frequently to calculate and print averages. \\
\verb'reset-averages' &  	boolean & 		false  &
Discard accumulated averages in restart file. \\ \hline
\verb'begin-rdf' &		integer & 		1000000 &
When to start accumulating radial distribution function information. \\
\verb'rdf-interval' &		integer & 		20 &
How frequently binning calculation is performed. \\
\verb'rdf-out' &		integer & 		5000 &
How frequently to calculate and print RDFs. \\
\verb'rdf-limit' &		real &			10 &
Calculate RDFs up to what distance? \\
\verb'nbins' &			integer & 		100 &
Number of binning intervals between 0 and rdf-limit. \\ \hline
\verb'dump-file' &	    	character  &	null &
Template of file names used for data dumps. \\
\verb'begin-dump' &		integer & 		1 &
Timestep to begin dumping at. \\
\verb'dump-interval' &  	integer & 		20 &
How frequently to perform dumps. \\
\verb'dump-level' &		integer & 		0 &
Amount of information to include in dump. \\
\verb'ndumps' &		integer & 		250 &
Number of dump records in each dump file. \\ \hline
\verb'backup-interval' &	integer & 		500 &
Frequency to write backup file. \\
\verb'backup-file' &		character  &	MDBACKUP &
Name of backup file. \\ \hline
\verb'temp-file' &		character  &	MDTEMPX &
Name of temporary file used for writing restart configurations. \\ \hline
\verb'subcell' &		real &			0 &
Size of sub cell to divide MD cell into for link-cell force
calculation. \\ \hline
\verb'seed' &			integer & 		1234567 &
Seed for random number generator. \\ \hline
\verb'page-width' &	    	integer & 		132 & 
Number of columns on output paper. \\
\verb'page-length' &   	integer & 		44 &
Number of lines on a page of output. \\ \hline
\verb'mass-unit' &	    	real &			1.6605655e-27 &
Unit of mass for system specification file. \\
\verb'length-unit' &    	real &			1e-10 &
Unit of length for system specification file. \\
\verb'time-unit' &		real &			1e-13 &
Unit of time for system specification file. \\
\verb'charge-unit' &		real &			1.6021892e-19 &
Unit of charge for system specification file. \\ \hline
\end{tabular}
\end{minipage}
\end{table}

\section{Setting up the System}
\subsection{System Specification}
\label{sec:sys-spec}

The information which describes to \moldy\  the system to be simulated
and the interaction potentials is contained in a file known as the
{\em system specification file}.  This may be presented to \moldy\  in
either of two ways: If the control file parameter \verb'sys-spec-file'
is null or absent, it is assumed to be appended to the end of the
control file.  Otherwise the it is read from the file whose name is
the value of \verb'sys-spec-file'.

This file is divided into two sections.  First is the description of
the molecules, atoms or ions, which is followed by the potential
functions.  As for the control file, the input is case independent and
free format, but line structured. Blank lines, spacing and comments
are ignored.

The physical description consists of a series of entries, one for each
molecular species, terminated by the keyword \verb'end'. The entry
for species $i$ should have the form
\begin{displaymath}
\begin{array}{lllllll}
\multicolumn{7}{l}{\mbox{{\it species-name}}_{i} \; N_{i}}  \\
id_{1} & x_{1} & y_{1} & z_{1}&  m_{1} & q_{1} & name_{1} \\
id_{2} & x_{2} & y_{2} & z_{2}&  m_{2} & q_{2} & name_{2} \\
\vdots & \vdots & \vdots & \vdots & \vdots & \vdots & \vdots \\
id_{n_{i}} & x_{n_{i}} & y_{n_{i}} & z_{n_{i}}& 
m_{n_{i}} & q_{n_{i}} & name_{n_{i}} \\
\end{array}
\end{displaymath}
where {\it species-name}$_{i}$ is the name of the molecule and $N_{i}$
is the number of molecules of that type in the system. Each molecule
has $n_{i}$ atoms, one for each line in that group and each kind of
atom is identified by a number $id_{i}$ (the site id) which will be
used to specify the appropriate potential parameters. Its co-ordinates
are $(x_{i},y_{i},z_{i})$, its mass is $m_{i}$, its charge is $q_{i}$
and its name is $name_{i}$.  See Appendix \ref{sec:examples} for some
sample system specification files.

If there is more that one atom of any type (in the system - not just
the same molecule) it is sufficient to identify it by its $id$ (and
the site co-ordinates!).  If $m_{i}$, $q_{i}$ or $name_{i}$ {\em are}
given they must agree exactly with the previous values or an error
will be signalled.

Site ids, masses and charges are all checked for `reasonableness'
and impossible values cause an error. The set of site ids does not
have to start at 1 or be contiguous, but since this may indicate a
mistake, a warning is issued.

Following the physical specification is the specification of the
potential functions. This takes the form
\begin{displaymath}
\begin{array}{llllll}
\multicolumn{4}{l}{\mbox{{\it potential-type}}} \\
i & j & p^{1}_{ij} & p^{2}_{ij} & \ldots & p^{r}_{ij}  \\
k & l & p^{1}_{kl} & p^{2}_{kl} & \ldots & p^{r}_{kl}  \\
\vdots & \vdots & \vdots & \vdots & \vdots & \vdots \\
m & n & p^{1}_{mn} & p^{2}_{mn} & \ldots & p^{r}_{mn}  \\
\mbox{\verb'end'} \\
\end{array}
\end{displaymath}
where {\it potential-type} is one of the keywords
\verb'Lennard-Jones', \verb'Buckingham' or \verb'MCY' to identify the kind of
potentials to be used,  $i, j, k, l, m, n$ are site ids and
$p^{\alpha}_{ij}$ is the $\alpha^{th}$ potential parameter between
sites $i$ and $j$.  There should be one line for each distinct pair of
site ids.  If any pair is omitted a warning is issued and the
parameter values are set to zero.

The meaning of the parameters for the currently defined potentials is
as follows:
\begin{description}
\item[Lennard-Jones] The potential is 
\[\phi(r_{ij}) = \epsilon((\sigma/r_{ij})^{12}) -
(\sigma/r_{ij})^{6}),\] and has two parameters, $\epsilon ( \equiv p^{1}_{ij})$
and $\sigma ( \equiv p^{2}_{ij})$, which occur on each line in that
order.  Note that the definition of epsilon differs by a factor of 4
from another definition in common usage. 
\item[Buckingham] This includes potentials of the Born-Meyer type and
has formula \[\phi(r_{ij}) = -A_{ij}/r^{6}_{ij} + B_{ij}exp(-C_{ij}r_{ij}).\]
The three parameters appear on each line in the order $A, B, C$.
\item[MCY] This type supports potentials of the form of the water
model of Matsuoka, Clementi and Yoshimine\cite{matsuoka:75}, 
\[\phi(r_{ij}) = A_{ij}exp(-B_{ij}r_{ij}) - C_{ij}exp(-D_{ij}r_{ij}),\]
and the four parameters appear on the line in the order $A, B, C, D$.
\end{description}
Other types of potential types may be easily added: see
appendix~\ref{sec:newpot}.

It is possible to specify the units in which these quantities are given
by means of the control file parameters \verb'mass-unit',
\verb'length-unit', \verb'time-unit' and \verb'charge-unit' 
(which are themselves specified in SI units). All quantities read from
the system specification file (dimensions as well as potentials) are
taken to be in those units. Their default values are amu, \AA, 0.1ps
and $q_{e}$, which means that the unit of energy is kJ mol$^{-1}$.  So
to read in \AA, amu and kcal mol$^{-1}$, specify
\verb'time-unit=4.8888213e-14'.

Once the system specification has been read in, all quantities are
converted to `internal' units: a.m.u., \AA, ps, and $\sqrt{
\mbox{ a.m.u. \AA}^{3} \mbox{ps}^{-2}/(4 \pi \epsilon_{0})}$.
The prototype molecule for each species is then shifted so that its
zero of coordinates lies on its centre of mass, and rotated into the
principal frame (polyatomics only).

\subsection{The Initial Configuration}

\moldy\  provides two methods of setting up an initial configuration.
By default the {\em skew start} method of section \ref{sec:skewstart}
is used to place the molecular centres of mass in a regular
arrangement which ensures molecular separation.  If there is more than
one species present, molecules of each are chosen randomly for each
site.  Molecular orientations are chosen randomly from a uniform
distribution.  This method has been found to work well for reasonably
small or fairly isotropic molecules and it is anticipated that it will
be the usual method of starting a simulation of the liquid state.   On
the other hand, if the constituent molecules are sufficiently large
and irregular, or if it is intended to simulate the solid state then
the {\em lattice start} method will be more appropriate.

This method is activated by setting the control parameter
\verb'lattice-start' to 1, and creates the initial configuration by
periodic replication of some crystalline unit cell.  In that case
\moldy\  expects to find, following the \verb'end' which terminates the
system specification, an initial configuration specification of the
following form:
\begin{displaymath}
\begin{array}{llllllll}
\multicolumn{8}{l}{a\;b\;c\;\alpha\;\beta\;\gamma\;n_{x}\;n_{y}\;n_{z}} \\
\mbox{{\it species-name}}_{1} &
X_{1} & Y_{1} & Z_{1} &  q_{10} & q_{11} & q_{12} & q_{13} \\
\vdots &  \vdots & \vdots & \vdots & \vdots & \vdots & \vdots & \vdots \\
\mbox{{\it species-name}}_{i} & X_{i} & Y_{i} & Z_{i} \\
\vdots &  \vdots & \vdots & \vdots & \vdots & \vdots & \vdots & \vdots \\
\mbox{{\it species-name}}_{n} &
X_{n} & Y_{n} & Z_{n} &  q_{n0} & q_{n1} & q_{n2} & q_{n3}  \\
\mbox{\verb'end'}. \\
\end{array}
\end{displaymath}
Here $a, b, c$ and $\alpha, \beta, \gamma$ are the crystal unit cell
parameters, and $n_{x}, n_{y}, n_{z}$ are the number of unit cells in
each direction which comprise the MD cell.  The next $n$ lines
describe the $n$ molecules of the basis which will be replicated to
form the full configuration.  Molecules may appear in any order, but
of course the total number of each, multiplied by the number of unit
cells $n_{x} n_{y} n_{z}$ must agree with that given in the system
specification file.

Each molecule is identified by its name, as given in the system
specification file.  $X, Y$ and $Z$ are {\em fractional} co-ordinates,
between 0 and 1 giving the location of the molecular centres of mass in
the crystal unit cell. The orientation is given by the four
quaternions $q_{0}, q_{1}, q_{2}, q_{3}$ which specify a rotation {\em
relative to the orientation of the prototype molecule in the system
specification file}.  (Notice the slight inconsistency with the
positions, which are of the centres of mass, {\em not} the zeroes of
co-ordinates in the system specification file. This may be fixed in
future releases.)  Quaternions need only be included for
polyatomic species, that is molecules $1$ and $n$ above, and omitted
for the monatomic species $i$.

After the molecular positions and orientations have been set up, their
velocities (and angular velocities if appropriate) are initialised.
Their values are sampled at random from the
Maxwell-Boltzmann distribution for the temperature $T$, as given by
the control parameter \verb'temperature'.  This is done for both
starting methods.

\section{Restarting from a Previous Run}
\label{sec:restarting}

At the end of a simulation run, it is often desirable to store the
configuration of the system in a file.  This {\em restart file} may be
used at a later date to continue the simulation from that point rather
than from scratch.  To instruct \moldy\  to write a restart file, simply
set the control parameter \verb'save-file' to a suitable filename; to
start from a restart file, set \verb'restart-file' to be the name of
that file.

Each restart file is a binary file which contains enough information
to reconstruct the exact state of the system and of the program.  It
includes a copy of all the control parameters in force, the current
timestep number, a complete system specification, all the simulation
dynamic variables and the intermediate data used in the calculation of
averages and radial distribution functions.  Thus a run continued from
a restart file will proceed just as if there had been no interruption
and will generate identical results (provided the control parameters
are not changed).

When continuing a simulation, it is only necessary to explicitly
specify control parameters which are to be changed.  Their previous
values are read from the restart file and are used as defaults when
reading the control file. Consequently control files for restarting
tend to be rather short. {\bf Caution}: always include a new (possibly
null) value for \verb'save-file'. Otherwise when the new run
terminates, the new restart file may overwrite the old one\footnote{
Whether the old file is lost depends on the operating system. Under
systems such as VMS and COS which have version numbers a new version
is created and the old one remains. Under Unix, the old file is
renamed by the addition of a ``\%'' character and thus is saved. On
other systems it will be lost.}.

Neither is it necessary to repeat the system specification since that
too is stored in the restart file.  However there are occasions
when it is desirable to do just that, for example if the value of one
of the potential parameters is to be modified.  In that case, set the
switch \verb'new-sys-spec' to 1 (true) and provide a system
specification as per a new simulation.  This is checked for
consistency with the existing one and if correct replaces it.  The
following checks are applied, which only verify that it is sensible to
assign the old dynamic variables to the new system. {\em 1.} The
number of species must be the same. {\em 2.} Each species must have
the same number of rotational degrees of freedom as its predecessor.
That is, it is not possible to replace a polyatomic by a monatomic or
linear molecule, for example.  {\em 3.} The number of molecules of
each species must not change.  This means that the order in the
specification file must be identical too.  It is however possible to
change the number of sites on a molecule, subject to {\em 2}.

\label{sec:backup}
Closely related to restarting is the backup mechanism.  This is
provided to guard against the complete loss of a simulation due to
computer failure. Periodically during a run, \moldy\  writes its state
to a {\em backup file} -- which is in fact just a restart file.  In
the event of a crash, the simulation can be restarted from the point
the last backup was written rather than from the beginning. The
related control parameters are \verb'backup-file' which specifies the
file name and \verb'backup-interval' which gives the frequency of
backups.  It should not normally be necessary to change the name, but
the optimum interval will depend on the size of the simulated system
and the speed of the computer. By default it is 500.  At the
successful end of a run the backup file is deleted so that only if
there is an abnormal termination does one remain\footnote{A backup
file is also written if the run is terminated for exceeding its cpu
limit.}.

The restart from a backup is entirely automatic. If a backup file
exists when a run is started, it is read in and the run continues from
it.  In contrast to a normal restart all of the control parameters are
taken from the backup file and the control file (and a restart file if
one is specified) is ignored\footnote{This is not quite true. \moldy\ 
does read the control file and any restart file but only to determine the
name of the backup file. Thus even if the backup has a non-standard name
it can still be found.}.

In consequence, if a run is aborted or stops abnormally for some
reason, the backup file must be removed ``by hand'' otherwise next
time a run starts, the unwanted simulation will continue instead.

A restart or backup file is created by first writing the data to a
temporary file which is then renamed to the final name.  This ensures
that there is no possibility of a file being left incomplete or
corrupt if the computer crashes part-way through the write.  If the
file already exists either it is replaced (on systems which only keep
one version of a file) or a new version is created (on systems such as
COS or VMS which retain multiple versions).  In the unlikely event of
it being necessary to change where the temporary file is
kept\footnote{This may be necessary if the restart file is located on
a different device or disk partition from the current directory. To
rename the temporary file successfully, it must reside in the same
partition or device as the restart file}, it may be specified with the
control parameter
\verb'temp-file'.

\section{Setting the Temperature}

To control the temperature of the simulated system, \moldy\  implements
the velocity rescaling technique described in section
\ref{sec:rescaling}.  Every \verb'scale-interval' timesteps until
\verb'scale-end', the velocities are adjusted so that the kinetic
energy corresponds exactly to the desired temperature (the value of
control parameter \verb'temperature').

There are several variants of scaling performed which are selected
by the control parameter \verb'scale-options'.  This is an integer
parameter interpreted as a set of bit flags with the following meanings.
\begin{description}
\item[bit 0]	perform scaling for each molecular species individually.
and potential energy.	      
\item[bit 1]	scale the rotational and translational components of
the kinetic energy separately.
\item[bit 2]	use the rolling averages of kinetic energy to
calculate the scale factor rather than the instantaneous values.
\item[bit 3]    discard all existing velocities and accelerations and
re-initialize from the Maxwell-Boltzmann distribution.
\end{description}
These bits may be set in any combination so, for example
\verb'scale-options=6' sets bits 1 and 2 ($ 6 = 2^1 + 2^2$) and scales
separately for rotation/translation using the rolling averages.  If
bit 3 is set the others are ignored.

The options for scaling separately rotation and translation, and per
species may be useful for achieving equilibration in ``difficult''
systems where mode-coupling is ineffective.  In those situations it is
otherwise possible for all the energy to be transferred into the
rotational modes of a particular species, halting any progress to
equilibrium for other degrees of freedom.   These options ensure that
all degrees of freedom have some thermal energy.

The option controlled by bit 3, to discard all existing information
and start from a random set of velocities may be of use when starting
from far-from-equilibrium situations.  In such cases the forces
are frequently so large that the velocities and accelerations exceed
the limits of the integration algorithm and timestep, which results in
\moldy\ stopping with a {\em quaternion normalization} or {\em
quaternion constraint} error.  Judicious use of this option every few
timesteps (using \verb'scale-interval') ought to allow the system to
relax to a state sufficiently close to equilibrium for normal scaling
to take over.

Bit 2 is intended to deal with the problem of setting the temperature
accurately using scaling.  The {\em ensemble average} kinetic energy
which characterizes the temperature of the system and the
instantaneous value fluctuates about this value.  However in the
traditional implementation of scaling, velocities are multiplied by a
factor of $\sqrt{\mbox{\em desired KE} / \mbox{\em instantaneous
KE}}$.  Thus the scaling factor is ``wrong'' by the ratio of the
instantaneous to average KE's which means that the temperature can not
be set more accurately than the relative size of the fluctuations in
the KE.  

The option selected by bit 2 goes some way towards the ideal scaling
factor by using the rolling average KE instead of the instantaneous
value.  The fluctuations in this short-term average should be much
lower than in the instantaneous value, allowing more accurate
temperature control. 

\section{Output}
\label{sec:output}
At the beginning of each run \moldy\  writes a {\em banner page}
containing a summary of the system being simulated and details of the
important control parameters. The bulk of the output file is the {\em
periodic output} which contains the instantaneous values of various
thermodynamic variables, their rolling averages and associated
standard deviations. The {\em rolling average} is just the mean over
the preceding $n$ timesteps where $n$ is set by the control parameter
\verb'roll-interval'.   An annotated example is given in
figure~\ref{fig:output}.  The frequency of periodic output may be
altered by setting the control parameter \verb'print-interval' to the
interval required.  (This may be necessary to constrain the size of
the output file which can grow to be very large indeed with the
default interval of only 10.)  
\begin{figure}
\caption[Sample \moldy\  output.]{Sample \moldy\  output from a simulation 
of a two component mixture.  The first component is a polyatomic
molecule and the second is atomic.  There are three {\em frames}, for
the instantaneous values, the rolling averages and their associated
standard deviations.  Within a frame, each row has the following
meaning: for translational and rotational kinetic energies and
temperatures it is the per-species value; for the potential energy it
is the direct and reciprocal space components, and the MD cell matrix,
{\tt h} and the stress are laid out as $3\times 3$ matrices.}
\label{fig:output}
\tiny
%
\begin{tabular}{rrrrrrrr
@{\hspace{1em}}r@{\hspace{1em}}rr@{\hspace{1em}}r@{\hspace{1em}}r}
\multicolumn{13}{l}{      Fri Nov 17 15:13:06 1989        Water\_test
Page 4} \\
Trans KE & Rot KE & Pot Energy & Tot Energy & TTemp & RTemp & Temp &
h(1,*) & h(2,*) & h(3,*) & Stress & Stress & Stress \\
\multicolumn{13}{l}{ ======== Timestep 10      Current values
======================================================== } \\
243.88 & 453.88  & -187.35  & 533.5  & 305.5  & 568.6  & 424.4  &
12.53  & 0.00  & 0.00  & 589  & 46.4  & 120 \\
22.053  & 0  & 1.0401 & & 221.0  & 0.0 & & 0.00  & 12.53  & 0.00  & 46.4
& 373  & 90.1 \\
 &  &  &  &  & & &  0.00  & 0.00  & 12.53 & 120  & 90.1  & -207 \\
\multicolumn{13}{l}{ -------- Rolling averages over last 10 timesteps
--------------------------------------------------------------------------
------------------------------------------------
} \\
240.27 & 319.31 & -82.472 & 533.39 & 301.0 & 400.0 & 342.9 & 12.53 &
0.00 & 0.00 & 1.2e+03 & 296 & 127 \\
22.077 & 0 & 34.205  & & 221.3 & 0.0 & & 0.00 & 12.53 & 0.00 & 296 &
589 & 133 \\
& & & & & & & 0.00 & 0.00 & 12.53 & 127 & 133 & -132 \\
\multicolumn{13}{l}{ -------- Standard deviations
---------------------------------------------------------------------------------------------------------------------------------------------------
} \\
1.8214 & 71.893 & 56.441 & .19942 & 2.3 & 90.1 & 43.4 & 0.00 & 0.00 &
0.00 & 1.32e+03 & 750 & 51 \\
0.013 & 0 & 17.173 & & 0.1 & 0.0 & & 0.00 & 0.00 & 0.00 & 750 & 119 & 55.2 \\
 & & &  &  &  &  & 0.00  & 0.00 & 0.00 & 51 & 55.2 & 49 \\
\end{tabular}
\end{figure}

As well as the ``short term'' rolling averages, long term averages are
calculated and printed out at regular but usually infrequent
intervals.  Accumulation starts on the timestep given by the control
parameter \verb'begin-average' and every \verb'average-interval'
timesteps thereafter, the means and standard deviations are calculated
and printed.  This output is interspersed with the periodic output and
is formatted with one variable to a line in the form {\em mean +/-
s.d.\ }. Where a variable has more than one component (such as
multiple species for the translational temperature or cartesian
components for the mean square forces) the components are printed
across the page\footnote{Remember that the standard deviation is a
measure of the {\em fluctuations} about the mean, {\bf not} the {\em
uncertainty} in the mean. For that the standard error in the mean is
required, which is more difficult to evaluate.  Theoretically it is
the {\em s.d.\ } divided by $\sqrt N$ where $N$ is the number of
independent observations.  But successive timesteps are highly
correlated and do not count as independent.}.  In addition to those
variables printed as part of the periodic output, the pressure, the
virial, mean square forces, mean square torques and total dipole
moments are calculated.

\section{Radial Distribution Functions}

Radial distribution functions are calculated by binning site pair
distances periodically throughout the simulation.  As this process is
expensive in computer time the binning subroutine is invoked only
every few timesteps, as set by the control parameter
\verb'rdf-interval' (20 by default).  Since the pair distances only
change a little on each timestep, very little statistical information
is lost.  Collection of binning data may also be
turned off during an equilibration period: specify when binning is to
start by means of the parameter \verb'begin-rdf'.  The parameters
\verb'rdf-limit' and \verb'nbins' control the details of binning,
giving respectively the largest distance counted and the number of
bins that interval is divided into.

Every \verb'rdf-out' timesteps (by default 5000) the rdfs are
calculated from the binned distances and printed out, and the counters
are reset to zero to begin accumulation again.  Distances are binned
and rdfs calculated separately for each distinct type of site-site
pair.  It is possible to calculate rdfs about an individual site,
distinguishing it from otherwise identical sites by assigning it a
unique site id in the system specification file.  An explanation of
the output format is given in figure~\ref{fig:rdf-output}.  Note that
each number should be considered as the value at the {\em centre} of
its bin, so that entry $i$ in each list is the value of
$g_{\alpha\beta}((i+1/2)b)$ where $b$ is the bin width.

\begin{figure}
\caption[Example output of radial distribution functions.]{Example
output of radial distribution functions. After the header line
consisting of underscores there is an indication of the bin width $b$
(that is the distance between points at which the rdf is tabulated).
Then for each site type pair $\alpha\beta$ there is a line listing
which pair (\eg {\tt O-O RDF}) followed by {\tt nbins}.  values of
$g_{\alpha\beta}((i+1/2)b)$.  }
\label{fig:rdf-output}
\begin{tiny}
\begin{verbatim}
---------------------------------------------------------------------------------------------------------------------------------------------------------------------------------
        Radial Distribution Functions   Bin width=0.1
        O-O RDF
 0.000000 0.000000 0.000000 0.000000 0.000000 0.000000 0.000000 0.000000 0.000000 0.000000 0.000000 0.000000 0.000000 0.000000
 0.000000 0.000000 0.000000 0.000000 0.000000 0.000000 0.000000 0.000000 0.000481 0.035710 0.183334 0.442186 0.613992 1.024402
 1.046396 0.964906 0.830174 0.660035 0.693341 0.615902 0.593192 0.510595 0.530697 0.532030 0.535959 0.524457 0.523221 0.466219
 0.496028 0.438487 0.456500 0.410547 0.443861 0.457956 0.446822 0.452202 0.419768 0.439333 0.465509 0.486887 0.461970 0.475745
 0.478883 0.480854 0.509090 0.533728 0.552747 0.552555 0.575402 0.547278 0.544836 0.493597 0.488168 0.520727 0.508073 0.479948
 0.501159 0.484000 0.485378 0.489160 0.464448 0.466791 0.476508 0.446576 0.470948 0.474468 0.449340 0.462169 0.501220 0.519107
 0.513338 0.510192 0.499766 0.525963 0.504663 0.517673 0.498359 0.512156 0.507061 0.466390 0.464342 0.445886 0.417555 0.407778
 0.387220 0.374041
        O-H RDF
 0.000000 0.000000 0.000000 0.000000 0.000000 0.000000 0.000000 0.000000 0.000000 26.976688 0.000000 0.000000 0.000000 0.000000
 0.000000 0.016214 0.061257 0.304082 0.647342 0.847404 0.757188 0.601222 0.478273 0.462682 0.449614 0.450424 0.518998 0.572242
 0.689704 0.914269 1.184674 1.441772 1.570390 1.609068 1.600392 1.430457 1.322722 1.183606 1.103701 1.061788 0.980018 0.960570
 0.924390 0.908883 0.877591 0.857668 0.890761 0.852463 0.815447 0.824963 0.841255 0.890416 0.929030 0.960589 0.984145 1.020650
 1.028199 1.047496 1.064600 1.099812 1.095715 1.073793 1.078131 1.049212 1.052160 1.052001 1.020737 1.010782 0.979748 0.983158
 0.988946 0.967620 0.955655 0.944384 0.952145 0.948509 0.946692 0.960097 0.959299 0.964074 0.969219 0.972704 0.998504 1.027791
 1.041576 1.037637 1.039961 1.016804 1.004726 1.026805 1.030903 1.006268 0.972421 0.948140 0.908959 0.877089 0.849855 0.817964
 0.776986 0.721485

\end{verbatim}
\end{tiny}
\end{figure}

\section{Dumping}

The dump facility is provided in order to allow the calculation of
dynamic properties, such as time correlation functions and additional
static averages not normally calculated by \moldy.  During a run, dump
files are produced which contain a record of the simulation dynamic
variables (positions, quaternions \etc) at varying degrees of
detail.  Any property of interest, dynamic or static, may then be
evaluated using the data in the dump.

A dump consists of a sequence of files since the amount of data
generated in a run can be very large indeed and it is usually more
convenient to manipulate a series of smaller files rather than one
large and unwieldy one. \moldy\  takes considerable pains to ensure that
a contiguous sequence of dump files is maintained and also ensures
that dumps from different runs are not accidentally intermixed.  There
is no requirement that a dump file be produced by a single run of
\moldy\ , which extends an existing file or starts a new one as appropriate.
A simulation may stop and be restarted many times without
disturbing the dump file sequence.  The  sequence
should (in most cases) even survive a system crash and a restart from
a backup file (see section~\ref{sec:backup}).

Each dump file in a sequence is a binary file consisting of a {\em
dump header}, which contains information about the contents of the
file followed by a number of {\em dump records} which contain the
actual data.

Several control parameters govern dumping. It starts at the timestep
specified by \verb'begin-dump', and a dump record is written every
\verb'dump-interval' timesteps thereafter.  After \verb'ndumps' dump
records have been written to a file, it is closed and another is
begun. Filenames are generated from a prototype (given by the
parameter \verb'dump-file') by appending a number, so that if the
prototype is {\em MDDUMP} then successive files will be named
{\em MDDUMP0}, {\em MDDUMP1}, {\em MDDUMP2} \etc
If it is not convenient for the sequence number to appear at the end
of the file, include the characters ``\%d'' at an appropriate
point\footnote{This is actually the code ``sprintf()'', the C library
function uses to signify converting an integer to a decimal character
string. This function is used to create the actual file name from the
prototype and the integer dump sequence number. (See any C library
manual for details.)}.  For example under VMS, specifying
\verb'dump-file=mddump%d.dat' will name the files {\em mddump0.dat},
{\em mddump1.dat} \etc.

Each dump record is a sequence of single-precision floating point
binary numbers.  Its exact contents are determined by the control
parameter \verb'dump-level' which is a bit flag {\em i.e.} a value
of $2^{n}$ means that bit $n$ is set.  Four bits are used and any
combination may be specified but the cumulative values 1, 3, 7
and 15 are most useful.  A value of 0 disables dumping. The data
dumped for each bit is as follows:
\begin{description}
\item[bit 0]	centre of mass co-ordinates, quaternions, unit cell matrix 
and potential energy.	      
\item[bit 1]	centre of mass velocities, quaternion and unit cell
matrix derivatives.
\item[bit 2]	centre of mass accelerations, quaternion and unit cell
matrix second derivatives.
\item[bit 3]    forces, torques and stress tensor.
\end{description}
Items selected are written in the order laid out above.  Within each
set of variables, values are ordered primarily by species in the order
they appeared in the system specification.  Within a species ordering
is by molecule (or atom) and at the finest level by $x$, $y$ or $z$
component ($q_{0}, \ldots q_{3}$ for quaternions). Therefore if $n$ is
the total number of molecules and $n_{r}$ is the the number with
rotational freedom the size of each record is
\begin{displaymath}
\begin{array}{cll}
   & 3n + 4n_{r} + 9 + 1 & \mbox{ (if bit 0 is set)} \\
 + & 3n + 4n_{r} + 9 & \mbox{ (if bit 1 is set)} \\
 + & 3n + 4n_{r} + 9 & \mbox{ (if bit 2 is set)} \\
 + & 3n + 3n_{r} + 9 & \mbox{ (if bit 3 is set)} \\
\end{array}
\end{displaymath}
single precision floating point numbers.

The header is a copy of a \verb'struct dump_t' (see
appendix~\ref{sec:structs} for the format). It contains the simulation
title and version number, the timestep at the beginning of the file,
the control parameters \verb'dump-interval' and \verb'dump-level', the
maximum and actual number of dump records in the file, a unique marker
(actually a timestamp), common to all the files in a dump run, and the
timestamp\footnote{A timestamp is simply the number of seconds elapsed
since midnight on January 1, 1970.} of any restart file used to start
the run.

Unfortunately it is not possible to dump directly to magnetic tape.  
\moldy\  must rewind to the beginning of a file to keep the
header up to date with the number of dumps in the file, as well as
extend existing files.  Neither operation is allowed on a tape drive.
Large disk stores are now very cheap so this should not be a problem
in practice.  If disk store {\em is} limited then the simulation may
be divided into multiple \moldy\  runs interspersed with copying of dump
files to tape.

Notice that \moldy\  must sometimes read an existing but complete dump
file to propagate the unique marker to all of the files in a sequence.
Therefore when continuing a simulation and a dump run, at least
the immediately preceding dump file must still be accessible.  This
should be borne in mind when copying dumps to tape!

\moldy\  is careful to ensure that existing files are not
overwritten - especially necessary since dump records are added to
the end of an existing dump file.  Whenever \moldy\  prepares to start a
new dump file it checks to see if one of that name is already present.
If so, a new name is chosen by ``mutating'' the old one, and a warning
message to that effect is written to the output file. On the other
hand, if the {\em first} file of a new dump run (including one
initiated because of some error in continuing an old one) already
exists, the {\em prototype} file name is mutated as above and the
whole dump run is written to files based on the mutated name.

When a run is restarted checks are made to ensure that the values of
the dump control parameters have not been altered.  If they
have, it is not possible to continue an existing dump sequence and a
new one will be started.  (If existing dump files are present the new
sequence will have mutated file names.)  This also happens if an
existing file does not appear to be a \moldy\  dump.  Existing dump
files are also tested to ensure that there is no corruption (due, for
example to a system crash) and that they contain the correct number of
records.  If the dump sequence can not be continued in these
circumstances, \moldy\  terminates with a fatal error rather than waste
computer time.

%To ensure that the header always records the correct number of dump
%records in a file, it is updated {\em after} the actual record has
%been written.  It is possible that following s crash there may be one
%more record than recorded.  \moldy\  ignores it and recalculates and
%dumps the data.

It is frequently convenient to perform analysis of dump data, and
perhaps graphical output on a different computer to that which
generated the data.  Unfortunately it is not usually possible to
sensibly transfer binary data between computers of different
architectures.  Therefore a program called {\em dumpconvert} is
provided which converts dump files to a portable text file format
(which may be easily moved between machines) and back again. It is
described in appendix~\ref{sec:dumpconvert}.

Two other utility programs included in the distribution are {\em
dumpanalyze} which identifies dump files by printing out the headers
and {\em dumpextract} which extracts atomic or molecular trajectories.
The latter should be useful as a prototype for writing programs to
analyse dump data.

\section{Constant Stress Simulation}

Setting  the control parameter   \verb'const-pressure' switches from a
constant-volume  simulation to a constant-stress  simulation using the
method     of    Parinello     and    Rahman\cite{parinello:81}   (see
section~\ref{sec:const-stress}).   The value of  the  ``MD cell'' mass
parameter,  $W$  is given by the control  parameter \verb'w'  and  the
external pressure by \verb'pressure'. At present it is not possible to
specify an anisotropic external stress, though this  capability may be
added in future versions of the program.

\section{Cutoffs and Adjustable Parameters}

There are four parameters related to the Ewald sum method of force
evaluation (see section~\ref{sec:ewald}).  The $\alpha$ parameter is
specified by \verb'alpha' and the direct and reciprocal space cutoff
distances $r_{c}$ and $k_{c}$ by \verb'cutoff' and \verb'k-cutoff'.
They are specified in units of \AA$^{-1}$, \AA\ and \AA$^{-1}$
respectively.  Before a new simulation is started, trial runs should
be performed to establish suitable values of these three parameters.
For a simulation of 512 MCY water molecules the values $\alpha =
0.3$\AA$^{-1}$, $r_{c} = 9$\AA\ and $k_{c} = 1.9$\AA$^{-1}$ give
potential energies correct to approximately 1 part in $10^{5}$. For a
simulation including ions - 1.1 Molal Magnesium Chloride solution -
the same accuracy is attained with $\alpha = 0.45$\AA$^{-1}$, $r_{c} =
9$\AA\ and $k_{c} = 3$\AA$^{-1}$.

The other relevant parameter is the switch \verb'surface-dipole' which
includes the dipole surface energy term of De Leeuw, Perram and
Smith\cite{deleeuw:80}.  See the note in section~\ref{sec:ewald} for an
explanation of why this term should {\em never} be used for an ionic
(as opposed to dipolar) system.

The one adjustable parameter which controls the link-cell force
calculation (see section~\ref{sec:link-cell}) is \verb'subcell'. It
specifies the length (in \AA) of the side of a link cell and determines
the number of cells the MD cell is divided into. In fact the MD cell
is divided into a whole number of subcells whose side in each of the
three directions is nearest to the value of \verb'subcell'.  (The
default of zero though, is special and sets subcell to one fifth of
the cutoff radius.) In general the smaller the link cell, the more
accurately the cutoff radius is implemented, but too many of them
reduces the efficiency of the program.

It is worth noting that it is unnecessary to recompile the program or
change anything else when the cutoffs are modified. Unlike most MD
codes, \moldy\  employs dynamic array allocation and automatically sets
up arrays of the correct size (and no more!) for any given $k_{c}$.
In the case of $r_{c}$ any value up to the length of the smallest MD
cell edge may be given. Any attempt to exceed this will result in the
program terminating with an error\footnote{In the unlikely event this
limit is too small, the modification is simple.  Simply set the {\tt
\#define}d constant {\tt NSH} in {\tt force.c} to the maximum number of
MD cell lengths required and recompile.  This will use slightly more
memory, but not unreasonably more}.

\section{Framework Simulations}

There has recently been much interest in simulations of systems of
molecules interacting with some rigid framework such as zeolites,
clays and other surfaces.  \moldy\  has the capability to include such a
framework in a simulation by defining it as a special kind of molecule.

The system specification should contain an entry, similar to that for
a normal molecule, which describes the atomic sites belonging to one
MD cell's worth of the framework.  Its periodic images should fill
space to construct the required infinite framework.  This is notified
to the program by modifying the first line of the specification of
that molecule to read
\begin{displaymath}
\begin{array}{lllllll}
\multicolumn{7}{l}{\mbox{{\it species-name}}_{i} {\tt \; 1 \; Framework}}  \\
\vdots & \vdots & \vdots & \vdots & \vdots & \vdots & \vdots \\
\end{array}
\end{displaymath}
(compare with section~\ref{sec:sys-spec}).  The effect of the special
keyword \verb'framework' is
\begin{enumerate}
\item to remove the rotational freedom of the molecule.  This
preserves the infinite structure over MD cell repeats by disallowing
relative motion of its parts. (Linear motion does not destroy the
structure and {\em is} allowed.)
\item to modify the effect of the periodic boundary conditions.
Normally a molecule is assumed to be ``small'' and periodic relocations
are applied to {\em all} of its atoms depending on its centre-of-mass
co-ordinates relative to some interacting molecule.  In contrast, the
atoms of a framework are independently relocated.  This ensures that
each molecule ``sees'' all framework atoms from any unit cell which
are within the cut-off distance.
\end{enumerate}

In the present version of the program, only one framework molecule is
allowed, though more may be permitted in future versions. Consequently
the configuration given as a lattice start must fill the entire MD
box. (A skew start is not sensible under these circumstances since the
orientation of the framework must be explicitly specified to construct
a good space-filling structure.)

\section{Messages and Errors}

Apart from the periodic output, there are occasional ``once-off''
messages which \moldy\  writes to the usual output file.  Such messages
begin with the characters \verb'*I*', \verb'*W*', \verb'*E*' or
\verb'*F*' denoting the classes {\em information}, {\em warning}, {\em
error} or {\em fatal} respectively.  Their meanings are
\begin{description}
\item[{\tt*I*}] information. These are often produced by subroutines
to give useful information on their particular calculations.  For
example when temperature scaling is turned off a message to that
effect is recorded in the output file.  Various routines which
calculate internal quantities such as the Ewald sum self energies and
distant potential corrections also record their values using an
information message.
\item[{\tt*W*}] warning.  When the system specification is suspicious
but not clearly wrong, or some untoward condition is detected such as
two atoms approaching too closely, a warning message is issued.
\item[{\tt*E*}] error. Error messages are issued when an mistake is
detected reading any of the input files. To make correction easier,
processing continues until the end of that file, so that all of the
errors are found.  The simulation is then stopped with a fatal error.
\item[{\tt*F*}] fatal. The simulation is terminated immediately. Faulty 
input files generate fatal errors after they have been completely
processed.  There are many other conditions which also generate fatal
errors, for example if the simulation algorithms violate some
criterion such as quaternion normalisation or constraints (see
section~\ref{sec:quaternions}), if the program runs out of memory or
if a restart file can not be correctly opened or is of the wrong
format.
\end{description}
The text of the messages is intended to be self-explanatory.

\chapter{Compiling and Modifying Moldy}
The \moldy\  distribution consists of numerous files of `C' source code
for \moldy\  and the utility programs, command or job files to compile
the source, \LaTeX input for the manual and example control and system
specification files.  For ease of transport these are packed into one
large ``archive'' file, whose format and method of unpacking depends
on the operating system of the target machine.  At present it is
available for:

\begin{description}
\item[unix]  The archive is either a tar archive called {\em
moldy.tar} or a Bourne shell script called {\em moldy.shar}. The first
is unpacked by the command \verb+tar xvf moldy.tar+ and the second by
\verb+/bin/sh moldy.shar+.

\item[VMS]  The archive is a DCL command file called {\em moldy.com},
and is unpacked by the command \verb'@moldy'.

\item[COS]  The archive is a JCL job file. It does not contain the
manual, examples or utilities, which are only of much use on the
front-end system.  N.B.  For unicos, use the unix distribution.
\end{description}


\section{Compilation}
The source code of \moldy\  consists of 20 files of C programs (which have
suffix ``.c'') and 4 header files (suffix ``.h'')\footnote{There is
also a supplementary file ``kernel.f'' which contains a FORTRAN
replacement for the function {\bf kernel()} in ``kernel.c''. This is
because certain poor C compilers (notably CRAY CC 4.1) are unable
to vectorize loops containing calls to the math library functions {\bf
sqrt()} and {\bf exp()}.  If your compiler suffers from this problem,
define the preprocessor symbol ``FKERNEL'' while compiling
``force.c''; this selects the FORTRAN version. Compile ``kernel.f''
and include it in the link {\em as well as ``kernel.c''}.  Both the
JCL file for COS and the makefile for unicos do this.}.
To build \moldy\ , all of the ``.c'' files must be compiled and linked
together to form an executable.  The method of doing this depends on
the operating system of the target computer.
\begin{description}
\item[unix]  The ``make'' program is used and the make file is
supplied in the distribution.  You will have to edit this file, called
``Makefile'' to select the  set of compiler flags appropriate for your
machine (see the next section for what to do if it is not catered
for). Then just type \verb'make' to compile and link \moldy\  and
\verb'make utilities' for the utility programs.

\item[VMS]  Simply type \verb'@compile' to execute the command file
``compile.com''. This will build \moldy\  and the utilities.

\item[COS]  In contrast to the other cases the cray COS distribution
is a self-contained JCL  file which, when executed, will compile and
link \moldy. 
\end{description}

\section{Portability}
A major goal in writing \moldy\ was that it be as portable as possible
between different computers  and operating systems.  It is  written in
the Kernighan and Ritchie\cite{kernighan:78} compatible subset of ANSI
C and assumes the library calls and header files  defined for a hosted
implementation of the  standard.  It  should therefore be possible  to
compile and run \moldy\ on any computer which has a good C compiler.

There are two sources of difficulty in moving \moldy\  to a new machine.
The library may not contain some necessary ANSI routine, or
may not have the required header files.  Secondly, to make good use of
vector or parallel architectures, compiler directives or calls to
specialized library functions are usually required.
Replacement ANSI library functions are supplied for the VAX/VMS,
unix (both Berkeley and AT\&T system V varieties) and COS operating
systems.  Different versions of a function are selected by the C
preprocessor and conditionally compiled according to the pre-defined
preprocessor symbols (see the documentation for your compiler).  For
ease of portability {\em all system-dependent functions are in the
module ``aux.c''} and {\em all preprocessor conditionals are in the
header file ``defs.h''}.

If the target machine has ANSI conformant C libraries, all that must
be done is to define the preprocessor symbol \verb+ANSI_LIBS+, either
in ``defs.h'' or by using a compiler option {\em e.g}.\
\verb+-DANSI_LIBS+.  This is done automatically in ``defs.h'' for
several machines known to have conformant environments.  If the target
operating system is the system V variant of UNIX, the preprocessor
symbol \verb'USG' is defined automatically in ``defs.h''.  It is
possible that some environments may defeat the selection making it
necessary to define it by hand, either in ``defs.h'' or by setting
the compiler option \verb'-DUSG' in the makefile. In addition, if the
symbol \verb'HAVE_VPRINTF' is defined, it is assumed that the library
contains the function \verb'vprintf()', otherwise the replacement in
``aux.c'' is used. If your system does not have \verb'vprintf()' but
does have \verb'_doprnt()' then define the symbol \verb'HAVE_DOPRNT'
to use it.

\subsection{System Dependencies}
In this section, details of system-dependant functions are described
for the major operating systems.  
\begin{itemize}
\item The ANSI header file ``string.h''	is missing
from Berkeley unix, and incomplete in other systems.  A file,
``string.h'' is supplied with \moldy\ which conditionally includes
``strings.h'', or ``string.h'' for other systems.  In addition the
ANSI header files ``stddef.h'' and ``stdlib.h'' are missing from many
unix systems and ``time.h'' does not contain the required
declarations.  Replacements are included  which again may
be dispensed with on an ANSI conformant system.
\item The  ANSI typedefs \verb'time_t' and \verb'size_t' are not
defined in some UNIX implementations. They are explicitly defined in
``time.h'' and ``stddef.h'' and the \verb'time()' function is
explicitly declared in ``aux.c''.
\item The supplied \verb'clock()' function on UNIX systems resets to
zero after 36 minutes.  Replacements, called \verb'cpu()' for both
system V and Berkeley UNIXes are supplied in ``aux.c''. (Otherwise
\verb'cpu()' simply calls \verb'clock()'.)
\item The ANSI function to delete a file, \verb'remove()' is missing
from UNIX systems and COS.  Replacements are supplied in ``aux.c''.
\item Replacements are provided for functions \verb'memset()' and
\verb'memcpy()' which are missing from Berkeley UNIX.
\item The function \verb'vprintf()' is often absent from older
libraries. Replacements are provided which {\em a)} call the internal
function \verb'_doprnt()' (or  \verb'XPRINTF()' in the case of COS) or
{\em b)} implements a portable \verb'vprintf()'.  Use the preprocessor
macros  \verb'HAVE_VPRINTF' or \verb'HAVE_DOPRNT' to select which.
\item ``Aux.c'' contains the functions \verb'replace()' and \verb'purge()'.
\verb'replace()' renames a file, making a backup of any existing file
of that name. \verb'purge()' removes the previous or backup version of
a file.  The pure ANSI versions just interface to \verb'rename()' and
do nothing respectively.
\end{itemize}

\subsection{Optimisation and Vectorisation}
\moldy\  has been designed to run fast on a wide range of computers, and
in particular on those with vector architectures. This is a difficult
problem, since necessarily machine-specific compiler directives and
optimisations are required for ultimate performance, and inevitably
portability must be sacrificed to gain speed.

Nonetheless, it has been found that following a few basic rules gives
extremely good performance on a range of vector computers.
\begin{itemize}
\item Write vector loops simply to make minimum demands on the
compiler's ability to vectorize.
\item Minimize the floating-point operation count.
\item Minimize memory references in vector loops.
\item Ensure that arrays are accessed with a stride of 1.  (This makes
a large difference on some machines, notably the Stellar, and little
on others). 
\item Minimize integer arithmetic in critical code.  The CRAY in
particular has no  integer multiplication hardware, which operations
are slow as a result.
\item Make use of manufacturer's linear algebra library functions.
\end{itemize}
The performance of \moldy\  has been carefully studied using profiling
tools, and all critical regions of code are written as efficiently
vectorizable loops.  

The most critical sections of code (\ie those which use the majority
of the computer time) are all to do with the site forces calculation.
Thus it is the inner loops in ``force.c'', ``ewald.c'' and
``kernel.c'' to which most attention should be paid.  The
pair-distance loop of \verb'rdf_calc()' in ``rdf.c'' should vectorize
for efficient radial distribution function evaluation. Others which
are of minor importance are in ``beeman.c'', ``matrix.c'',
``quaterns.c'' and ``algorith.c''.  ``Aux.c'' contains alternative
versions of various sum, dot product, scatter and gather routines
\etc which are interfaces to machine-specific libraries \eg
Cray scilib, Convex veclib (which usually have FORTRAN calling
conventions). There are also default versions coded in C which do
vectorize, for machines lacking specialist libraries as well as for
scalar computers.

The program should, of course, be compiled with options specifying
vectorisation. Many compilers also have an option which directs it to
ignore recurrences throughout the whole program, \eg \verb'-va' on the
Convex, \verb'-va' and \verb'+CaliasFreePointers' for the Stellar and
\verb'-h ivdep' on the Cray compilers. It should  normally be safe to
use these options.  Since highly optimizing and vectorizing compilers
frequently contain bugs, and since some options generate ``unsafe''
optimizations,  it may be necessary to restrict the highest
optimisation level to those modules which contain critical code.

To  allow  the compiler  to  generate   vector code,  machine-specific
compiler  directives must be inserted  into the  source. The reason is
that the  run-time  dimensioned arrays necessary   to implement such a
flexible  program   must use   pointers as  their  base.    (See any C
textbook, \eg    Kernighan  and  Ritchie\cite{kernighan:78}    for  an
explanation of C pointers  and arrays.)  Unfortunately this means that
the  compiler  can not determine  that each iteration  of  the loop is
independant of the preceding iterations.  In the jargon of vectorizing
compilers, there may be a {\em vector dependency} or {\em recurrence}.
Each manufacturer's compiler has its own  peculiar method to inform it
that despite appearances there is no dependency.  For example the CRAY
compilers use  a  \verb'#pragma'~\verb'ivdep'   statement  whereas the
convex and Stellar compilers use a ``significant comment''
\verb'/*$dir'~\verb'no_recurrence*/'.


A mechanism is provided to insert appropriate directives using the C
preprocessor.  The text \verb'VECTORIZE' has been placed before each
loop which ought to be vectorized, and the file ``defs.h'' contains
machine-conditional \verb'#define's to replace it with the appropriate
directive.  Currently directives for the CRAY, Stellar and Convex
compilers are included, and null text is substituted for other
machines.  Notice that in each case the substituted text is {\em not}
the directive described in the manual, but rather that directive {\em
after} it has been passed through the preprocessor. To determine what
should be substituted on a new vector machine, create a small test
containing the documented directive and use the C preprocessor on that
file. The output will show the form that should be defined in
``defs.h''.\footnote{Note that there is a bug in the Ardent (now
Stardent) compiler whereby the \verb'\#pragma'~\verb'ivdep' directive acts
syntactically like a C statement.  There are several loops in
``force.c'' and ``ewald.c'' where the body of an \verb'if' clause is a
single \verb'for' loop without braces.  The directive forces the
\verb'for' loop {\em outside} the scope of the \verb'if' clause,
erroneously altering the program flow.  The solution is to modify the
code by putting braces around the directive and the \verb'for' loop.}

This mechanism only works for vectorising compilers which have a
directive which means ``ignore dependencies in the following loop''.
Some, notably the Alliant FX/C compiler take the alternative approach
of having direcives to the effect of ``ignore dependencies involving
the following variables''.  In that case the directives must be
inserted ``by hand'', though even then the \verb'VECTORIZE' directive
al least identifies the appropriate loops.

Unfortunately this method had been obsoleted by the ANSI C standard,
which makes it impossible to insert directives using the preprocessor.
With more modern compilers such as Cray scc the same effect may be
achieved using ``ignore dependencies'' command-line directives.

\section{Modifying Moldy}
\subsection{Adding a New Potential}
\label{sec:newpot}
By default \moldy\  supports potential functions of the Lennard-Jones,
six-exp and MCY forms.  However it should be very easy to add further
types. The program is written in a highly modular fashion so that {\em
the only code which need be altered is in file ``kernel.c''} (and
occasionally in ``defs.h''). 

The calculation of the potential and forces is performed entirely in
the function \verb'kernel()'.  This function is called repeatedly with
a vector of (squared) distances between some reference site and its
neighbour sites.  Vectors of potential parameters and charges are
supplied which bear a one to one correspondance with the elements of
the distance vector.  It calculates the coresponding values of
${dU(r_{ij})} \over {dr_{ij}}$ which it stores in \verb'forceij[]'.
There are several variants of the force-calculation loop, one for each
kind of potential.  The potential type in use is passed as a parameter
to \verb'kernel()' and is used in a \verb'switch' statement to select
the appropriate code. 

To add a new potential the array of structs called \verb'potspec[]'
must be extended.  The new array element should contain the name of
the new potential (against which, the names given in system
specification files will be matched) and the number of potential
parameters for each site pair\footnote{By default up to five
parameters are allowed. If this is not sufficient, the limit is set by
the value of the constant \verb'NPOTP' defined in ``defs.h''}.  Then
define a new preprocessor symbol to the index of the new type in the
array \verb'potspec[]' (after the line
\verb'#define'~\verb'MCYPOT'~\verb'2'.  The value must correspond to
the index of the new entry in \verb'potspec[]' (starting from 0 in
accordance with the usual C convention).  This constant should be used
to define a new case in the \verb'switch' statement of
\verb'kernel()', and this is where the code to evaluate the potential goes.

The  existing  cases  may  be  used as  a    model, especially for the
evaluation of the electrostatic term $\mbox{erfc}(\alpha r) / r$ which
is    evaluated  by  the   polynomial  expansion    of  Abramowitz and
Stegun\cite[section 7.1.26]{abramowitz:70}. There  are  currently {\em
two}   versions of each    loop,  the   second omitting this  term for
efficiency when  all the  electric  charges  are  zero  (which case is
flagged by a negative value of $\alpha$).

Finally, the distant potential correction for the new potential should
be added as a new case to function \verb'dist_pot()'. The code should evaluate
\[
\int^{\infty}_{r_c} r^2 U(r) dr
\]
for the potential $U(r)$.
\subsection{Parallel Version}
An experimental parallel version of \moldy\  is under development, and
currently runs on a Stellar GS1000.   It consists of replacements for
files ``force.c'' and ``ewald.c''. 

The approach taken is to retain the vectorized inner loops and
parallelize the the outer loops, over cells for ``force.c'' and {\bf
k} vectors for ``ewald.c''.  The two parallel loops are structured by
the method developed by the Kingston group\cite{wojcik:85}. That is, a
loop of the form
\begin{verbatim}
	for(i = 0; i < N; i++)
	   <body of loop>
\end{verbatim}
is transformed into
\begin{verbatim}
	for(processor = 0; processor < NPROCESSORS; processor++)
	   for(i = processor; i < N; i += NPROCESSORS)
	      <body of loop>
\end{verbatim}
so that the outer loop is over the number of available processors,
and the inner loop has an increment of that number.  Note that this
transformation reorders the loop iterations.

If the outer loop is to execute concurrently, each sub-process or
thread must have local copies of any temporary variables, notably
\verb'i'.  Therefore the contents of each loop are isolated into a
separate function. Each thread executes a separate incarnation
of the function.  A shared-memory model is assumed, in which each
thread accesses a single copy of the parameters and global
variables, but has a unique copy of all of its local variables.

The ``master'' thread maintains separate copies of result variables
and arrays, (\ie the forces, potential energy and stress virial) so
that each incarnation of the function can write to its own copy
without interference from other threads.  After all of the threads
recombine, the separate copies are summed to give the final result of
the calculation.  This method avoids the need for explicit
synchronization between threads, but will add to the serial overhead
on highly parallel machines.

The method of actually parallelizing the outer loop will vary from
machine to machine. The stellar compiler is able to generate parallel
code, and the directive \verb'/*$dir'~\verb'no_recurrence*/' instructs
it to parallelize the following loop.  On other machines this may not
be so easy!
\appendix
\chapter{Example System Specifications}
\label{sec:examples}
\section{Argon}
\begin{verbatim}
# LJ Argon - about as simple as you can get
Argon 128
1          0           0          0    36 Ar
end
Lennard-Jones
1 1 ? ?
\end{verbatim}
\section{TIPS2 Water}
This is the four-site water model of Jorgenson 
\cite{jorgensen:82}.  Only the oxygen site interacts via the Lennard-Jones
potential, and the charge site, M, is displaced 0.15\AA from the Oxygen.
\begin{verbatim}
# Modified TIPS2 water
Water 64
1 	   0 		0 	   0 16    0 O
2  0.7569503		0 -0.5858822 1 0.535 H
2 -0.7569503		0 -0.5858822
3          0            0      -0.15 0 -1.07 M
end
lennard-jones
1 1 0.51799  3.2407
end
\end{verbatim}
\section{Aqueous MgCl$_2$ Solution}
This is a three-component system consisting of MCY
water\cite{matsuoka:75}, Magnesium and Chloride ions.  The Mg$^{2+}$
potential was fitted to the SCF calculations of Dietz and
Heinzinger\cite{dietz:82} and the Cl$^-$ to the calculations of
Kistenmacher, Popkie and Clementi\cite{kistenmacher:73b}.  Note that
the potential parameters are expressed in kcal mol$^{-1}$, and the
control file must set the parameter \verb'time-unit=4.8888213e-14'.
\begin{verbatim}
# MCY Water/ Mg2+ / Cl - solution
Water 200
1 	   0 		0 	   0 16    0 O
2  0.7569503		0 -0.5858822 1     0.717484 H
2 -0.7569503		0 -0.5858822
3          0            0    -0.2677 0 -1.434968 M
Magnesium 4
4	   0		0	   0 24.31 2 Mg2+
Chloride  8
5	   0		0	   0 35.45 -1 Cl-
end
MCY
1 1 1088213.2 	5.152712 	0 		0
1 2 1455.427 	2.961895 	273.5954 	2.233264
2 2 666.3373 	2.760844 	0 		0
1 4 47750.0	3.836		546.3		1.253 # New values of Mg potl
2 4 111.0	1.06		0		1.0   
1 5 198855.0	3.910		0		0
2 5 1857.0	2.408		77.94		1.369
4 5 28325.5	2.65		0		0
end
\end{verbatim}
\chapter{Organisation of Moldy}
\section{Data Structures}
\label{sec:structs}
\section{Control Flow}

\chapter{Utility Programs}
\section{Moldyextract}
\section{Dumpanalyze}
\section{Dumpconvert}
\label{sec:dumpconvert}
\section{Dumpextract}
\section{Mdshak}

\bibliography{simulation,mypubs}
\bibliographystyle{unsrt}

\end{document}
